\subsection{It is not Microsoft Word}
LaTeX is completely free to use! The same can not be said about Microsoft
Word.

\subsection{There are tons of packages!}
With LaTeX, you can install hundreds of packages to use with your document.
This gives you a great degree of customization that isn’t found with Word processors. The possibilities are endless,``All you need is a box, and imagination.''


\begin{figure}[htp]
    \centering
    \includegraphics[scale=0.17]{imagination.png}

\end{figure}

\subsubsection{Package Examples}
There are some useful packages for LaTeX that you might want to learn later.
Make sure you make the package title bold when typing this list!

\begin{itemize}
\item \textbf{amsmath} (Making matrices)
\item \textbf{biblatex} (Citations/Bibliography)
\item \textbf{bookstab} (Nicer Tables)
\item \textbf{xcolor} (Add Color to Documents)
\end{itemize}

\subsection{Typing Code}
Since LaTeX is often used in mathematics and science, there is an easy way to
type code. There are a few different ways to achieve this, but here we are using
\underline{verbatim} like this\ldots

\begin{verbatim}
    System.out.println("Hello World");
\end{verbatim}

\subsection{Mathematics Features}
Probably the best feature of LaTeX is the ease in creating mathematical expressions. This is likely the reason why it is so widespread in the mathematics and
computer science research community.
